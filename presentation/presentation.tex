% OWD 2023
% Placement Interview Presentation for 'owd-euses'
%
% To generate the man page PDF in the listings/ directory:
%   $ man -t manual.1 > manual.ps
%   $ ps2pdf manual.ps

\documentclass{beamer}

% BEGIN COMPANY INFORMATION

\newcommand\companyname{Thales}
\date{27th September, 2023} % Date of the placement interview

% END COMPANY INFORMATION

\usepackage{minted, calc}

\usetheme{CambridgeUS}
\setbeamertemplate{caption}[numbered]
\numberwithin{figure}{section}
\beamerdefaultoverlayspecification{<+->}
\titlegraphic{\includegraphics[width=.25\textwidth]{./gentoo-horizontal}}
\newcommand\chext{.{\fontfamily{cmr}\{}c,h{\fontfamily{cmr}\}}}

\newcounter{firstsrcline}
\newcounter{lastsrcline}
\newcommand\srclinelimit{29}
\newcommand\resetsrclines{%
    \setcounter{firstsrcline}{1}
    \setcounter{lastsrcline}{\thefirstsrcline+\srclinelimit}
}
\newcommand\continuesource[1]{%
    \inputminted[firstline=\thefirstsrcline, lastline=\thelastsrcline,
        firstnumber=\thefirstsrcline]{c}{listings/#1}
    \setcounter{firstsrcline}{\thelastsrcline+1}
    \setcounter{lastsrcline}{\thefirstsrcline+\srclinelimit}
}
\renewcommand\theFancyVerbLine{\color{gray}\slshape\arabic{FancyVerbLine}}
\setminted{
    fontsize=\tiny,
    style=vs,
    linenos,
    numbersep=8pt,
    xleftmargin=12pt,
}

\hypersetup{
    colorlinks,
    allcolors=.,
    urlcolor=gray,
}

\newcommand\programname{\texttt{owd-euses}}
\title[Placement Interview Presentation]{The \programname\ Querying Tool}
\author[Oliver Dixon]{
    \texorpdfstring{Oliver Dixon \raisebox{.5pt}{%
        \texttt{\footnotesize\href{mailto:Oliver Dixon <od641@york.ac.uk>}
            {<od641@york.ac.uk>}}
    }}{Oliver Dixon}}
\institute[University of York]{Department of Computer Science, University of
    York \and Department of Mathematics, University of York}
\subtitle{\companyname\ Placement Interview Presentation}

\begin{document}
\frame{\titlepage}
\section{Introduction, Motivation, and Project Specification}
\begin{frame}
    \frametitle{Overview of the Environment: Gentoo Linux}
    \begin{itemize}
        \item Gentoo Linux is a source-based distribution of the Linux operating
            system; such distributions typically include package managers that
            encourage, or mandate, the local compilation of most packages. This
            stands in contrast to the previously established model of
            downloading and installing pre-built binaries.
        \item A source-based distribution model has a number of advantages for
            power users, namely the ability to enable and disable compile-time
            features as desired, resulting in a more compact and suitably
            equipped binary.
        \item With such extensive customisability, a powerful querying system is
            required to determine the various features that are supported by
            each of the available packages.
    \end{itemize}
\end{frame}
\begin{frame}
    \frametitle{Existing Solutions}
    Due to the age and popularity of Gentoo Linux, a multitude of querying tools
    exist for this purpose:
    \pause

    \begin{itemize}
        \item \emph{eix}: very powerful and well-established, but requires
            building and maintaining a centralised binary cache of every
            installable package.
        \item \emph{euses}: one of the original cornerstones of the Gentoo
            toolset, but contains bugs, error-prone and non-portable code, and
            is incompatible with newer versions of Gentoo.
        \item \emph{portageq}: the standard Gentoo querying tool; fully
            compliant, but proves to be very non-performant on large software
            repositories due to its antiquated Python foundations.
    \end{itemize}
\end{frame}
\begin{frame}
    \frametitle{Project Requirements}
    \programname\ was devised in an effort to unify the advantages of these
    existing solutions while eliminating the problems relating to safety,
    incompatibility, and performance-preserving scalability. The final tool must
    possess certain critical attributes:
    \pause

    \begin{enumerate}
        \item be fully standards-compliant with the Gentoo \emph{Package Manager
            Specification} (PMS) and sufficiently customisable to work on future
            revisions;
        \item operate on existing (plain text) files in the established
            directory structures, instead of generating and periodically
            maintaining binary caches;
        \item seamlessly scale on very large software repositories;
        \item be \emph{correct}: undergoing rigorous unit-testing,
            stress-testing under hostile environments, and never entering an
            undefined or non-deterministic state.
    \end{enumerate}
\end{frame}
\section{Technical Overview}
\begin{frame}
    \frametitle{Details of the Implementation and Source Structure}
    Written in C99, \programname\ was intended to run on POSIX-compliant
    Gentoo/PMS-conforming UNIX-like systems. Given the four major design
    principles, the codebase is very compact, currently standing at slightly
    over 2000 lines. The source is clearly sectioned into distinct functions and
    well-abstracted with regard to near-complete separation of code and data.
    \pause

    \begin{center}
        \begin{tabular}{ll}
            \hline
            \textbf{Translation Unit(s)} & \textbf{Purpose} \\
            \hline
            \texttt{euses\chext} & Provide the core parsing and searching
                routines \\
            \texttt{globbing\chext} & Utilise POSIX regex support \\
            \texttt{args\chext} & Parse and manage command-line arguments \\
            \texttt{stack\chext} & Implement the standard stack ADT \\
            \texttt{converse\chext} & Manage status- and error-reporting \\
            \texttt{colour.h} & Provide helpers for ANSI text-colouring \\
            \hline
        \end{tabular}
    \end{center}
\end{frame}
\begin{frame}
    \frametitle{Source Preview: Stack Interface (1)}
    \resetsrclines
    \continuesource{stack.h}
\end{frame}
\begin{frame}
    \frametitle{Source Preview: Stack Interface (2)}
    \continuesource{stack.h}
\end{frame}
\begin{frame}
    \frametitle{Source Preview: Stack Implementation (1)}
    \resetsrclines
    \continuesource{stack.c}
\end{frame}
\begin{frame}
    \frametitle{Source Preview: Stack Implementation (2)}
    \continuesource{stack.c}
\end{frame}
\begin{frame}
    \frametitle{Source Preview: Stack Implementation (3)}
    \continuesource{stack.c}
\end{frame}
\begin{frame}
    \frametitle{Usage Preview: Basic Search}
\end{frame}
\section{Deployment and Distribution}
% \section{Future Work}
%
% Note to self: keep this out of the presentation; the interviewers will likely
% ask follow-up questions regarding potential improvements. Such questions can
% be easily answered: add regex support for match queries, add wide character
% support, threading the search algorithms, optimising for shorter needle
% lengths, etc.
\section{Questions}
\begin{frame}
    \frametitle{Any questions?}
\end{frame}
\end{document}

